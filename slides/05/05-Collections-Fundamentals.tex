\documentclass[presentation]{beamer}
\usepackage{oop-slides}
\setbeamertemplate{bibliography item}[text]
\newcommand{\lessonnr}[0]{05}
\title[OOP\lessonnr{} -- Collections and Generics I]{\lessonnr{}\\ Javadoc, jar, polimorfismo parametrico e introduzione al Java Collections Framework}

\begin{document}
	
\frame[label=coverpage]{\titlepage}

%====================
%Outline
%====================
\newcommand{\al}[0]{\textless}
\newcommand{\ar}[0]{\textgreater}
\newcommand{\gen}[1]{\al{}#1\ar{}}
\newcommand{\imgfr}[4]{\fr{#1}{#2
\begin{center}
\includegraphics[width=#3\textwidth]{#4}                    
\end{center}
}}



\section{Esercitazione di oggi}
\fr{Preparazione Ambiente di Lavoro}{
  \iz{
    \item Accendere il PC
    \item Loggarsi sul sito del corso
    \iz{
      \item \textcolor{blue}{\moodleurl{}}
    }
    \item Scaricare dal sito il file \texttt{lab05.zip} contenente il materiale dell'esercitazione odierna
    \item Spostare il file scaricato sul Desktop
    \item Decomprimere il file usando 7zip (o un programma analogo) sul Desktop
    \item Importare la cartella \texttt{lab05} come progetto all'interno di Eclipse
  }
}


\section{Introduzione e utilizzo dei jar}
\subsection{I JAR file}
\fr{Java: i file JAR}{
  \iz {
    \item Finora, abbiamo visto che un'applicazione Java è composta di un insieme di classi: noi vorremmo distribuirla come singolo file!
    \item Normalmente, le applicazioni Java vengono confezionate in file JAR
  }
  \bl{Cos'è?} {
    \iz {
      \item Un JAR è un archivio (normalmente compresso) che contiene le classi, le risorse (e.g. icone) e un file descrittivo detto ``Manifest''
      \item Il Manifest viene creato sempre in \texttt{META-INF/MANIFEST.MF}, e contiene informazioni sull'applicazione, ad esempio su quale classe contenga il main del programma
      \item È possibile associare (a livello di sistema operativo) l'esecuzione del JAR file al comando Java, in modo che l'applicazione si avvii automaticamente ``col doppio click'' (avviando automaticamente\ la classe scritta nel Manifest)
      \item È possibile utilizzare i file JAR come componenti di altre applicazioni
    }
  }
}

\subsection{Gestione dei JAR file da Eclipse}
\fr {Generazione di file JAR con Eclipse} {
  Eclipse è in grado di supportare l'utente nella generazione di file JAR.
  \iz {
    \item Si seguano i passi nelle slide successive per generare un Jar a partire dai tre packages:
     \iz{
     	\item \texttt{it.unibo.oop.lab05.packages.pkg1} 
        \item \texttt{it.unibo.oop.lab05.packages.pkg2}
	\item \texttt{it.unibo.oop.lab05.packages.pkg3}
     }
  }
}

\imgfr{Generazione di file JAR con Eclipse}{Selezionare le risorse da esportare, click destro ed \texttt{Export}}{0.3}{img/exportjar1}
\imgfr{Generazione di file JAR con Eclipse}{}{0.6}{img/exportjar2}
\imgfr{Generazione di file JAR con Eclipse}{}{0.6}{img/exportjar3}
\imgfr{Generazione di file JAR con Eclipse}{}{0.6}{img/exportjar4}
\imgfr{Generazione di file JAR con Eclipse}{Espandere l'elenco dei file}{0.4}{img/exportjar5}
\imgfr{Generazione di file JAR con Eclipse}{Verificare \textbf{sempre} che siano selezionate \textbf{tutte e sole} le risorse che ci interessano.}{0.5}{img/exportjar6}
\imgfr{Generazione di file JAR con Eclipse}{Eventualmente potete esportare anche i sorgenti}{0.5}{img/exportjar7}
\imgfr{Generazione di file JAR con Eclipse}{Inserite il nome del file (e selezionate la cartella dove volete creare il JAR)}{0.5}{img/exportjar8}
\imgfr{Generazione di file JAR con Eclipse}{Finish}{0.5}{img/exportjar9}

\fr {Verifica del contenuto di un JAR da riga di comando} {
       Si verifichi il contenuto del JAR file con \texttt{jar tf myjar.jar}
  }

\fr {Generazione di file JAR eseguibili con Eclipse} {
  Eclipse è anche in grado di generare automaticamente il Manifest file corretto per mettere in esecuzione una certa classe.
  \iz {
    \item È necessario che esista già una ``Run configuration''. Quando lanciate una classe con \texttt{main} per la prima volta, questa viene creata automaticamente
    \item Si seguano i passi nelle slide successive per esportare i tre packages compilati, specificando la main class da eseguire
    %\item Si tenti di eseguire il JAR con \texttt{Java -jar myjar.jar}. Cosa succede?
  }
}

\imgfr{Generazione di file JAR eseguibili con Eclipse}{}{0.99}{img/exportrunnablejar1}
\imgfr{Generazione di file JAR eseguibili con Eclipse}{}{0.3}{img/exportrunnablejar2}
\imgfr{Generazione di file JAR eseguibili con Eclipse}{}{0.4}{img/exportrunnablejar3}
\imgfr{Generazione di file JAR eseguibili con Eclipse}{Selezionare le risorse di interesse}{0.99}{img/exportrunnablejar4}
\imgfr{Generazione di file JAR eseguibili con Eclipse}{}{0.3}{img/exportrunnablejar5}
\imgfr{Generazione di file JAR eseguibili con Eclipse}{}{0.7}{img/exportrunnablejar6}
\imgfr{Generazione di file JAR eseguibili con Eclipse}{Selezionare la la ``Run configuration'' da utilizzare}{0.7}{img/exportrunnablejar7}
\imgfr{Generazione di file JAR eseguibili con Eclipse}{Inserire il nome del file (e selezionare il path in cui si desidera crearlo)}{0.7}{img/exportrunnablejar8}
\imgfr{Generazione di file JAR eseguibili con Eclipse}{Finish}{0.7}{img/exportrunnablejar9}


\section{Scrivere e Generare la Documentazione di Programmi Java}

\fr{Javadoc: Introduzione 1/2}{
  \bl{Il ruolo fondamentale della documentazione}{
    La documentazione di un progetto software è un aspetto fondamentale
    \iz{
      \item Al fine di garantirne la manutenibilità
        \iz{
          \item Siamo così sicuri di ricordarci a distanza di settimane del perché abbiamo adottato una certa architettura, scritto una certa classe, una data estensione di una classe, un certo metodo?
          \item Arriva in azienda un nuovo sviluppatore: come fa a costruirsi il background necessario per lavorare su un progetto software esistente?
        }
      \item Al fine di aumentare la comprensione del codice
        \iz{
          \item Cosa farà mai il metodo \cil{do_stuff()} sviluppato dal collega?
          \item Ci sono dei metodi che sono disponibili solo per ragioni di compatibilità e che non dovrei più utilizzare?
        }
    }
  }
                                  
  \bl{}{
    Oggi ci concentreremo sulla \alert{Javadoc} che, integrata all'uso di UML, è una delle parti fondamentali di una buona documentazione di un progetto SW
  }
}

\fr{Javadoc: Introduzione 2/2}{
  \bl{Javadoc: che cos'è?}{
    Tool di supporto per la generazione automatica della documentazione (HTML-based) dei programmi Java, tramite utilizzo di annotazioni \textit{``speciali''} collocate in punti ben precisi dei sorgenti
  }
  
  \bl{Come funziona}{
    \iz{
      \item Il tool processa tutti i commenti del tipo \cil{/** ... */}
      \item Commenti che si trovano in testa a dichiarazione di classi, metodi, etc. vengono inclusi per generare automaticamente la documentazione
      \item Custom tag consentono di classificare diverse tipologie di informazioni, facilitando la scrittura e la generazione della documentazione
    }
  }
}

\fr{Javadoc: Principali Tag 1/2}{
  \iz{ 
    \item \textcolor{blue}{\cil{@param}}
      \iz{
        \item Utilizzabile nei commenti relativi a costruttori/metodi o in classi parametriche
        \item Descrive un generico parametro di input oppure un tipo generico
      }
  }
  \iz{ 
    \item \textcolor{blue}{\cil{@return}}
      \iz{
        \item Utilizzabile nei commenti relativi ai metodi
        \item Descrive il valore di ritorno
      }
  }
  \iz{ 
    \item \textcolor{blue}{\cil{@deprecated}}
      \iz{
        \item Utilizzabile nei commenti relativi a classi/costruttori/metodi/campi
        \item Descrive che quel particolare costruttore/metodo etc. è stato deprecato. E' ancora disponibile per ragioni di compatibilità ma è opportuno non utilizzarlo nello sviluppo di nuove applicazioni
      }
  }
}

\fr{Javadoc: Principali Tag 2/2}{
  \iz{ 
    \item \textcolor{blue}{\cil{@throws}}
      \iz{
        \item Utilizzabile nei commenti relativi a costruttori/metodi che lanciano una eccezione (le vedremo in seguito)
        \item Descrive l'eccezione e il motivo per cui viene lanciata
      }
  }
  
  \iz{ 
    \item \textcolor{blue}{\cil{@link}}
      \iz{
        \item Descrive collegamenti ipertestuali (link) a metodi/classi/costanti della stessa classe o anche di classi esterne
        \item Esempi di utilizzo
          \iz{
            \item \texttt{ \{@link package.OtherClass\#someMethod\} }
            \item \texttt{ \{@link \#someMethodOfSameClass\} }
            \item \texttt{ \{@link \#someFieldOfSameClass\} }
          }
      }
  }    

  \iz{ 
    \item \textcolor{blue}{\cil{@author}} ed \textcolor{blue}{\cil{@version}}
      \iz{
        \item Utilizzabili nel commento che descrive la classe
        \item Descrive l'autore/autori della classe, e la sua versione
        \item \textcolor{red}{Non usateli}. Esistono modi migliori (che vedremo in futuro in questo corso) per sapere esattamente chi ha fatto cosa, quando, e qual è la versione della classe.
      }
  }
}

\fr{Javadoc: Linee Guida d'Uso}{
  \bl{Cosa commentare e quali tag usare?}{
    \iz{
      \item Inserire sempre un commento (anche corto) che descrive il ruolo e il funzionamento generale della classe
      \item Inserire un commento per tutti i costruttori (parametri e return value inclusi), metodi, e campi con livello di accesso \cil{public} e \cil{protected}
      \item Non è necessario documentare metodi di cui si fa override, a meno che non vi siano peculiarità non rilevate nella documentazione della superclasse. In questo caso, è sufficiente utilizzare \textcolor{red}{\texttt{\{@inheritDoc\}}} all'interno del commento, e verrà generata Javadoc prendendola dalla superclasse.
    }
  }
  \bl{}{
    Utilizzare i sorgenti che vi forniamo in lab come linea guida di riferimento
  }
}

% \fr{Generazione Javadoc da Riga di comando}{
%   Esempio base di generazione della documentazione\\
%   
%   \iz{ 
%     \item {\tiny\texttt{javadoc -d <dest-dir> -sourcepath <src-path> -subpackages <base-pckg> -exclude <ex-pckg> -author}} 
%       \iz{
%         \item \textcolor{blue}{-d} directory in cui viene salvata la doc generata
%         \item \textcolor{blue}{-sourcepath} il path in cui trovare i sorgenti (da la possibilità di invocare il comando da qualunque cartella)
%         \item \textcolor{blue}{-subpackages} lista di package root da usare come base per generare la documentazione. Il comando andrà in ricorsione in tutti i sotto package
%         \item \textcolor{blue}{-exclude} lista di sotto package da escludere
%         \item \textcolor{blue}{-author} include il tag author alla doc generata
%       }
%   }
%   
%   Generiamo la documentazione per il progetto OOP1415-LAB (supponendo di trovarci in OOP1415-LAB)
%     \iz{
%       \item {\small\texttt{javadoc -d doc -sourcepath src -subpackages oop1415 -author}} 
%     }  
% }


\fr{Generazione Javadoc da Eclipse 1/2}{
  \fg{height=0.9\textheight}{img/javadoc1} 
}

\fr{Generazione Javadoc da Eclipse 2/2}{
  \fg{height=0.8\textheight}{img/javadoc2}
}

\fr{Javadoc: Esempio di Documentazione Generata}{
  \fg{height=0.8\textheight}{img/javadoc_view}
}


\fr{Modalità di Lavoro}{
  \bl{}{
    \en{
      \item Gli esercizi sono divisi in package con nomi progressivi
      \item Troverete un commento con le istruzioni per ciascun esercizio
      \item Risolvere l'esercizio in autonomia
      \item Cercare di risolvere autonomamente eventuali piccoli problemi che possono verificarsi durante lo svolgimento degli esercizi
      \item \textcolor{red}{Utilizzare le funzioni di test presenti nei sorgenti per il testing dell'esercizio}
      \item Contattare i docenti nel caso vi troviate a lungo bloccati nella risoluzione di uno specifico esercizio
      \item \textbf{Una volta ultimato l'esercizio chiamare i docenti per un controllo della soluzione}
       \item \textcolor{red}{Scrivere il Javadoc per l'esercizio svolto} 
      \item Proseguire con l'esercizio seguente
    }
  }
}

\end{document}


\subsection{I JAR file}
\fr{Deployment di applicazioni Java}{
  \iz {
    \item Finora, abbiamo visto che un'applicazione Java è composta di un insieme di classi: noi vorremmo distribuirla come singolo file!
    \item Normalmente, le applicazioni Java vengono confezionate in file JAR
  }
  \bl{Cos'è?} {
    \iz {
      \item Un JAR è un archivio (normalmente compresso) che contiene le classi, le risorse (e.g. icone) e un file descrittivo detto ``Manifest''
      \item Il Manifest viene creato sempre in \texttt{META-INF/MANIFEST.MF}, e contiene informazioni sull'applicazione, ad esempio su quale classe contenga il main del programma
      \item È possibile associare (a livello di sistema operativo) l'esecuzione del JAR file al comando Java, in modo che l'applicazione si avvii automaticamente ``col doppio click'' (avviando automaticamente\ la classe scritta nel Manifest)
      \item È possibile utilizzare i file JAR come componenti di altre applicazioni
    }
  }
}

\fr{Il comando \texttt{jar}}{
  Per creare un JAR, si utilizza il comando \texttt{jar}
  \bl{Opzioni importanti di \texttt{jar}}{
    \iz {
      \item \textbf{\texttt{c}} --- Specifica l'intenzione di creare un JAR file
      \item \textbf{\texttt{f}} --- Specifica un file di output (se non presente, l'output è rediretto su standard output)
      \item \textbf{\texttt{m}} --- Specifica l'intenzione di allegare un manifest file personalizzato (se non presente, ne viene creato uno di default, che non specifica alcuna classe da eseguire)
    }
  }
}

\fr{Uso di \texttt{jar}}{
  \bl{Esempi d'uso:}{
    \texttt{jar cf jar-file.jar file1 file2 directory1}
    \iz {
      \item Crea un nuovo JAR file di nome \texttt{jar-file.jar} contenente \texttt{file1} \texttt{file2} \texttt{directory1}. Include un Manifest di default
    }
    \texttt{jar cf jar-file.jar *}
    \iz {
      \item Crea un nuovo JAR file di nome \texttt{jar-file.jar} contenente tutti i files e le directory nel path corrente. Include un Manifest di default
    }
    \texttt{jar cmf manifest.mf jar-file.jar *}
    \iz {
      \item Crea un nuovo JAR file di nome \texttt{jar-file.jar} contenente tutti i files e le directory nel path corrente. Utilizza il file \texttt{manifest.mf} come file Manifest
    }
  }
}

\fr{File Manifest}{
  \bl{Esempio di Manifest}{
    \sizedcode{\scriptsize}{code/MANIFEST.MF}
  }
  \iz {
    \item versione del Manifest
    \item Class Path (vedremo fra poco)
    \item Main Class da eseguire
  }
}

\fr{Esecuzione di JAR file}{
  \iz {
    \item \texttt{java} ha un'opzione che consente l'esecuzione di file jar.
    \item Tale opzione è \texttt{-jar}
    \item Quando si lancia \\ \texttt{java -jar nomefile.jar}, la Java Virtual Machine automaticamente legge il file Manifest, cerca una descrizione della Main Class da eseguire e tenta di eseguirla.
  }
}

\fr{Primo esercizio con \texttt{jar}} {
  Facendo riferimento alle ultime slides, si svolga quanto segue:
  \iz {
    \item Ci si posizioni nella cartella \texttt{bin}
    \item Si crei un jar file contenente le classi appena compilate, utilizzando \texttt{jar cf} e le opportune opzioni.
    \item Si verifichi con il comando \texttt{jar tf nomefile.jar} il contenuto del jar. Esso dovrà corrispondere a quello della prossima slide.
  }
}

\fr{Soluzione} {
  \texttt{META-INF/ \\
  META-INF/MANIFEST.MF \\
  oop1415/lab05/packages/ \\
  oop1415/lab05/packages/pkg3/ \\
  oop1415/lab05/packages/pkg3/UseDummies.class \\
  oop1415/lab05/packages/pkg1/ \\
  oop1415/lab05/packages/pkg1/MinionDummy.class \\
  oop1415/lab05/packages/pkg1/DummyClass.class \\
  oop1415/lab05/packages/pkg2/ \\
  oop1415/lab05/packages/pkg2/EvenMoreDummy.class \\
  oop1415/lab05/packages/pkg2/MoreDummy.class}
}

% \fr{Altri esercizi con JAR} {
%   Facendo riferimento alle ultime slides, si svolga quanto segue:
%   \iz {
%     \item Ci si posizioni nella cartella \texttt{bin}
%     \item Si crei un jar file \textbf{eseguibile} contenente le classi appena compilate, utilizzando \texttt{jar cfm} e le opportune opzioni. Il file Manifest da utilizzare è fornito nella cartella \texttt{res/lab05/TestManifest.MF}
%     \iz {
%       \item Si ricorda che, per risalire nell'albero delle directory, è possibile utilizzare \texttt{../} \\ Ad esempio, \\ jar mcf ../res/lab05/Manifest.MF testfile.jar directory \\ crea un nuovo JAR prendendo il manifest dalla cartella superiore alla corrente in \texttt{res/lab05/Manifest.MF}
%     }
%     \item Si lanci con \texttt{java -jar} il JAR appena creato
%   }
% }

\fr {Generazione di file JAR con Eclipse} {
  Eclipse è in grado di supportare l'utente nella generazione di file JAR.
  \iz {
    \item Si seguano i passi nelle slide successive per esportare i tre packages compilati
    \item Si verifichi il contenuto del JAR file con \texttt{jar tf myjar.jar}
    \item Si tenti di eseguire il JAR con \texttt{Java -jar myjar.jar}. Cosa succede?
  }
}

\imgfr{Generazione di file JAR con Eclipse}{Selezionare le risorse da esportare, click destro ed \texttt{Export}}{0.3}{img/exportjar1}
\imgfr{Generazione di file JAR con Eclipse}{}{0.6}{img/exportjar2}
\imgfr{Generazione di file JAR con Eclipse}{}{0.6}{img/exportjar3}
\imgfr{Generazione di file JAR con Eclipse}{}{0.6}{img/exportjar4}
\imgfr{Generazione di file JAR con Eclipse}{Espandere l'elenco dei file}{0.4}{img/exportjar5}
\imgfr{Generazione di file JAR con Eclipse}{Verificare \textbf{sempre} che siano selezionate \textbf{tutte e sole} le risorse che ci interessano.}{0.5}{img/exportjar6}
\imgfr{Generazione di file JAR con Eclipse}{Eventualmente potete esportare anche i sorgenti}{0.5}{img/exportjar7}
\imgfr{Generazione di file JAR con Eclipse}{Inserite il nome del file (e selezionate la cartella dove volete creare il JAR)}{0.5}{img/exportjar8}
\imgfr{Generazione di file JAR con Eclipse}{Finish}{0.5}{img/exportjar9}


\fr {Generazione di file JAR eseguibili con Eclipse} {
  Eclipse è anche in grado di generare automaticamente il Manifest file corretto per mettere in esecuzione una certa classe.
  \iz {
    \item È necessario che esista già una ``Run configuration''. Quando lanciate una classe con \texttt{main} per la prima volta, questa viene creata automaticamente
    \item Si seguano i passi nelle slide successive per esportare i tre packages compilati, specificando la main class da eseguire
    \item Si tenti di eseguire il JAR con \texttt{Java -jar myjar.jar}. Cosa succede?
  }
}

\imgfr{Generazione di file JAR eseguibili con Eclipse}{}{0.99}{img/exportrunnablejar1}
\imgfr{Generazione di file JAR eseguibili con Eclipse}{}{0.3}{img/exportrunnablejar2}
\imgfr{Generazione di file JAR eseguibili con Eclipse}{}{0.4}{img/exportrunnablejar3}
\imgfr{Generazione di file JAR eseguibili con Eclipse}{Selezionare le risorse di interesse}{0.99}{img/exportrunnablejar4}
\imgfr{Generazione di file JAR eseguibili con Eclipse}{}{0.3}{img/exportrunnablejar5}
\imgfr{Generazione di file JAR eseguibili con Eclipse}{}{0.7}{img/exportrunnablejar6}
\imgfr{Generazione UML del modellodi file JAR eseguibili con Eclipse}{Selezionare la la ``Run configuration'' da utilizzare}{0.7}{img/exportrunnablejar7}
\imgfr{Generazione di file JAR eseguibili con Eclipse}{Inserire il nome del file (e selezionare il path in cui si desidera crearlo)}{0.7}{img/exportrunnablejar8}
\imgfr{Generazione di file JAR eseguibili con Eclipse}{Finish}{0.7}{img/exportrunnablejar9}

\subsection{Il classpath}
\fr{Il classpath}{
  \iz{
    \item Il classpath è l'insieme delle directory in cui Java cerca classi e packages.
    \item Sia \cil{java} che \cil{javac} consentono di specificarlo con l'opzione \cil{-cp}
  }
  \bl{In \cil{javac}}{
    Il classpath indica dove verranno cercate le classi necessarie affinché il programma sia compilato.
    \iz{
      \item Uso di librerie esterne
    }
  }
  \bl{In \cil{java}}{
    Il classpath indica dove verranno cercate le classi necessarie affinché il programma sia eseguito correttamente
    \iz{
      \item Uso di librerie esterne
      \item Software esportato come JAR file
      \item Risorse da caricare incluse in un JAR file
    }
  }
}

\fr{Compilazione: librerie mancanti}{
  \iz{
    \item Ci si posizioni con un terminale nella cartella contenente il progetto
    \item Si svuoti la cartella \texttt{bin} se presente, la si crei altrimenti
    \item Si provi a compilare la classe oop1415.lab05.mathlib.TestMathLab (utilizzando \texttt{javac} senza l'opzione \texttt{-cp} come in precedenza, il comando \textbf{dovete} saperlo a questo punto.)
    \item Cosa succede?
  }
  \bl{Leggere e capire gli errori. \textbf{Sempre}.}{
    \texttt{
      src/oop1415/lab05/mathlib/TestFastMath.java:6: error: package org.apache.commons.math3.util does not exist
    }
  }
}

\fr{Compilazione: aggiunta di librerie al class path}{
  \iz{
    \item Motivo: il programma si appoggia ad una libreria esterna (Apache Commons Math \cite{apache-commons-math})
    \item È necessario istruire javac dicendogli dove trovare la libreria
    \item Essa vi viene fornita in \texttt{lib/commons-math3-3.2.jar}
    \item Si ricompili, aggiungendo l'opzione \texttt{-cp lib/commons-math3-3.2.jar}
    \item La compilazione va a buon fine!
  }
}

\fr{Esecuzione: librerie mancanti}{
  \iz{
    \item Lo stesso problema si pone al momento dell'esecuzione
    \item Dato che avete solo compilato, ma non ``impacchettato'' un JAR che includesse tutto il necessario, è necessario dire alla Java Virtual Machine dove recuperare le classi necessarie per \textbf{eseguire} il programma.
%     \item Si noti che, in generale, le librerie richieste a tempo d'esecuzione possono essere diverse da quelle necessarie a tempo di compilazione (in genere, di più).
    \item Ci si posizioni con il terminale dentro la cartella \texttt{bin}.
    \item Si esegua la classe appena compilata con \texttt{java}, senza usare l'opzione \texttt{-cp} (anche stavolta, dovete essere in grado di scrivere correttamente il comando).
    \item Cosa succede?
  }
  \bl{Leggere e capire gli errori. \textbf{Sempre}.}{
    \scriptsize\texttt{
This program will run 1000000 calculations of asin and acos for each library, and print the time. \\
Java Math took 51.9713719ms to complete. \\
Exception in thread "main" java.lang.NoClassDefFoundError: \\ org/apache/commons/math3/util/FastMath \\
        at oop1314.lab05.mathlib.TestFastMath.main(TestFastMath.java:52)
    }
  }
}

\fr{Esecuzione: aggiunta di librerie al class path}{
  \iz{
    \item Il programma esegue finché possibile, quindi lancia un errore!
    \item La Java Virtual Machine, \textbf{a run time}, non riesce a caricare la classe \texttt{org.apache.commons.math3.util.FastMath}
  }
  Inclusione della libreria nel classpath
  \iz{
    \item Ci si posizioni con un terminale nella cartella del progetto
    \item Sappiamo che le nostre classi compilate stanno in \texttt{bin}, mentre la libreria necessaria in \texttt{lib/commons-math3-3.2.jar}
    \item Si provi ad eseguire da questa posizione, utilizzando l'opzione \texttt{-cp} di \texttt{java}:
    \iz{
      \item Su sistemi UNIX: \texttt{-cp bin:lib/commons-math3-3.2.jar}
      \item Su sistemi Windows: \texttt{-cp bin;lib/commons-math3-3.2.jar}
      \item Ossia: su UNIX i vari componenti del classpath si separano con i due punti, su Windows con il punto e virgola.
    }
  }
}

\fr{Gestione del Classpath da Eclipse}{
  \iz {
    \item Eclipse consente di aggiungere le librerie desiderate al classpath
    \item Si importi il progetto odierno in Eclipse
    \item Si seguano le istruzioni seguenti per includere nel classpath il file jar necessario alla compilazione
  }
} 

\imgfr{Gestione del Classpath da Eclipse}{Properties}{0.4}{img/eclipsejar1}
\imgfr{Gestione del Classpath da Eclipse}{Java Build Path}{0.8}{img/eclipsejar2}
\imgfr{Gestione del Classpath da Eclipse}{Libraries}{0.8}{img/eclipsejar3}
\imgfr{Gestione del Classpath da Eclipse}{Add JARs...}{0.8}{img/eclipsejar4}
\imgfr{Gestione del Classpath da Eclipse}{I JAR file devono essere dentro uno dei progetti}{0.6}{img/eclipsejar5}
\imgfr{Gestione del Classpath da Eclipse}{Normalmente, si posizionano in \texttt{lib}}{0.6}{img/eclipsejar6}
\imgfr{Gestione del Classpath da Eclipse}{}{0.6}{img/eclipsejar7}
\imgfr{Gestione del Classpath da Eclipse}{}{0.6}{img/eclipsejar8}
\imgfr{Gestione del Classpath da Eclipse}{}{0.8}{img/eclipsejar9}
\imgfr{Gestione del Classpath da Eclipse}{Appare ``Referenced Libraries'' nel progetto}{0.99}{img/eclipsejar10}
\imgfr{Gestione del Classpath da Eclipse}{All'interno, trovate il JAR e tutte le classi in esso contenute!}{0.99}{img/eclipsejar11}


% \subsection{Javadoc personalizzato: doclets}
% 
% \fr{Uso di doclet}{
% Doclets are programs written in the JavaTM programming language that use the doclet API to specify the content and format of the output of the Javadoc tool \cite{doclet}
% 
% 
% }

\section{Java collections framework e Java generics}

\fr {Istruzioni}{
  \iz {
    \item Nell'esercitazione di oggi troverete sorgenti parzialmente implementati
    \item Direttamente dentro i sorgenti, troverete una serie di punti da svolgere
    \item Per aiutarvi a trovarli, essi sono segnalati con il commento \cil{//TODO} ed un numero.
    \item Eseguite gli esercizi tenendo conto della numerazione
  }
}

%\imgfr{Struttura delle classi del model}{UML del modello}{0.99}{img/class}

\subsection{Completamento di classi che usano collections}

\fr {Utilizzo di Set\gen{E}}{
  \iz {
    \item Si completi la classe \texttt{oop1314.lab05.exams.model.Exam}.
    \item È richiesto l'uso dell'interfaccia Set, come implementazione è consigliato HashSet.
    \item Per dettagli sui metodi forniti dall'interfaccia, si consulti la Javadoc delle API Java
  }
}


\subsection{Costruzione di una classe generica}

\fr {Utilizzo di List\gen{E}: creazione di una memoria associativa}{
  \iz {
    \item Si osservi l'interfaccia \cil{oop1415.lab05.exams.generics.IAssociativeMemory<K, V>}
    \item Essa rappresenta una semplice memoria associativa, ossia una memoria in grado di associare due oggetti. Dato un oggetto, è possible risalire all'altro.
    \iz {
      \item Una (banale, vedrete \textbf{molto} di meglio a lezione) strategia implementativa è l'utilizzo di due liste ordinate.
      \item Quando una nuova associazione viene creata, gli oggetti vengono salvati in due liste diverse nella medesima posizione.
      \item Quando uno dei due oggetti viene rimosso, viene rimosso anche l'altro
      \item Quando viene richiesto un oggetto passando l'altro, si cerca in che posizione sia nella lista, quindi si ritorna l'elemento nella stessa posizione dell'altra lista
    }
    \item Tenendo conto di quanto detto, si realizzi la classe \cil{oop1415.lab05.exams.generics.AssociativeMemory<K, V>}, che implementa l'interfaccia \cil{oop1415.lab05.exams.generics.IAssociativeMemory<K, V>}.
  }
}

\subsection{Ordinamento con Comparable\gen{T}}

\fr {Implementazione di Comparable\gen{T}}{
  \iz {
    \item Si implementi \cil{Comparable<IPerson>} in \cil{IPerson}.
    \item Si implementi in \cil{AbstractPerson} il metodo necessario a supportare \cil{Comparable<IPerson>}. Tale metodo deve ordinare le persone alfabeticamente  per cognome.
    \item Si ricorda che la classe \cil{String} implementa già \cil{Comparable<String>}: le stringhe hanno ordine naturale uguale all'ordine alfabetico.
  }
}

\fr {Test}{
Si testi il funzionamento di quanto fatto terminando l'implementazione di \texttt{oop1415.lab05.exams.ExamManagerTest}
}

\subsection{Uso di wildcard bounded}
\fr {Varianza}{
  \iz {
    \item Si rimuavano i commenti la prima parte commentata in \cil{PeopleManagerTest}
    \item Errori!
    \item Ragioniamo su cosa stiamo facendo: il metodo vuole prendere una \cil{Collection<IPerson>} per aggiungerla internamente al \cil{PeopleManager}
    \item Noi stiamo passando una \cil{Collection<Student>}.
    \item Dato che \cil{Student} implementa \cil{IPerson}, ci aspetteremmo che le \cil{Collection<Student>} siano anche \cil{Collection<IPerson>}, giusto?
    \item \textbf{NO.}
    \item Infatti, ad una \cil{Collection<IPerson>} possono essere aggiunti anche dei \cil{Professor}, mentre ad una \cil{Collection<Student>} no.
  }
}

\fr {Non ci servono le add: uso di \cil{? extends}}{
  \iz {
    \item Nel nostro caso, però, l'uso sarebbe lecito: usiamo la \cil{Collection<Student>} per \textbf{accedere} ai suoi elementi, ma non li aggiungiamo, e l'operazione di aggiungere degli \cil{Student} ad un \cil{IPeopleManager<IPerson>} è lecita.
    \item In Java, è possibile imporre delle limitazioni per dire al compilatore che utilizzeremo quell'oggetto solo per \textbf{accedere} ai suoi elementi
    \item Si modifichi il metodo\\ \cil{void addAllFromCollection(Collection<PersonType>)} in: \\ \cil{void addAllFromCollection(Collection<? extends PersonType>)}
    \item Si osservi che gli errori spariscono
    \item Si tenti di inserire un nuovo elemento nella \cil{Collection<? extends PersonType>} passata: cosa succede?
    \item Il compilatore sa che la \cil{Collection} contiene un sottotipo di \cil{PersonType}, ma non sa quale: non vi consente di aggiungere elementi.
  }
}

\fr {Contro-varianza}{
  \iz {
    \item Si decommenti la seconda parte commentata in \cil{PeopleManagerTest}
    \item Errori!
    \item Ragioniamo su cosa stiamo facendo: il metodo vuole prendere una \cil{Collection<IPerson>} per aggiungerci il contenuto del \cil{PeopleManager}
    \item Noi stiamo passando una \cil{Collection<Object>}.
    \item Dato che ogni tipo Java è sottotipo di \cil{Object}, ci aspetteremmo che le \cil{Collection<IPerson>} siano anche \cil{Collection<Object>}, giusto?
    \item \textbf{NO.}
    \item Infatti, gli elementi di una \cil{Collection<Object>}, non sempre sono delle \cil{IPerson}!
  }
}

\fr {Non ci servono le get: uso di \cil{? super}}{
  \iz {
    \item Nel nostro caso, però, l'uso sarebbe lecito: aggiungiamo dei PersonType alla \cil{Collection<Object>}, e questo è perfettamente lecito
    \item In Java, è possibile imporre delle limitazioni per dire al compilatore che utilizzeremo quell'oggetto solo per \textbf{aggiungere} degli elementi, ma non ne controlleremo il contenuto
    \item Si modifichi il metodo\\ \cil{void addAllToCollection(Collection<PersonType>)} in: \\ \cil{void addAllToCollection(Collection<? super PersonType>)}
    \item Si osservi che gli errori spariscono
    \item Si tenti di prendere elemento della \cil{Collection<? super PersonType>} passata: cosa succede?
    \item Il compilatore sa che la \cil{Collection} contiene un supertipo di \cil{PersonType}, ma non sa quale: vi ritorna un Object.
  }
}

\section{Prova d'esame}
\subsection{Progettazione di una classe generica}
\fr{Esercizio con classe generica}{
  \iz {
    \item Si legga e si risolva l'esercizio in \texttt{UseThree.java}.
    \item Per problemi, suggerimenti o richieste, si chieda e ci si aiuti anche fra studenti facendo uso del forum del corso.
  }
}
\subsection{Costruzione di un iteratore}
\fr{Esercizio con iteratore}{
  \iz {
    \item Si legga e si risolva l'esercizio nel file testo.txt in \texttt{oop1415.lab05.examsaple2}.
    \item Per problemi, suggerimenti o richieste, si chieda e ci si aiuti anche fra studenti facendo uso del forum del corso.
  }
}

\begin{frame}[allowframebreaks]
 \frametitle{Bibliography}
   \bibliographystyle{plain}
   \small
 \bibliography{biblio}
\end{frame}
